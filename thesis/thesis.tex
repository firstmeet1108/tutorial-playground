% !TEX encoding = UTF-8 Unicode
% !TEX TS-program = xelatex
% !BIB program = biber

\documentclass{HDU-Bachelor-Thesis}

% 导入 \addbibresource \printbibliography
\usepackage{biblatex}
% 导入 \includepdf
\usepackage{pdfpages}
% 导入 \sout
\usepackage{ulem}
% 导入 \setmathfont
\usepackage{unicode-math}
% 导入 \textcolor
\usepackage{xcolor}

\title{基于web container的前端教学站点的设计与实现}
\author{靳一鸣}
\date{2025 年 3 月 1 日}
\HDUyear{2025}
\HDUschool{计算机学院}
\HDUmajor{软件工程}
\HDUclassid{21052711}
\HDUstudentid{21011114}
\HDUadviser{舒亚非}
\HDUfinishdate{2025 年 4 月}
\HDUsigndate{2025 年 4 月 5 日}

% 从 ref.bib 中读入参考文献
\addbibresource{ref.bib}

% 设置公式字体(其它字体在模板中已设置好)
\setmathfont{texgyretermes-math.otf}

\begin{document}
% 页码在 PDF 阅读器中显示为罗马数字
\pagenumbering{roman}
% 关闭页眉、页脚
\pagestyle{empty}

% 生成封面
\maketitle
% 或者使用 Microsoft Word 制作封面后导入
%\includepdf[pages={-}]{封面.pdf}

\clearpage
% 关闭页脚
\pagestyle{HDU-bachelor-empty}

\section*{摘\hspace{2em}要}

随着互联网技术的不断发展,各种技术和工具不断出现,开发者们学习技术和工具的成本也越来越高;由最初的服务端渲染到如今的客户端渲染乃至同构渲染,人们从最初的学习一门语言变成了如今的需要学习一整套工具链,开发者们开始疲于学习新的框架和技术。

本毕业设计旨在设计并实现一个基于 web container 的前端教学站点,方便教师使用文档和向导的方式步一步地重现开发过程,同时学生可简单修改示例代码,进行交互式验证,整个过程不需要架设任何后端。本文围绕Web Container的实现与应用展开,详细介绍了Web Cotainer的实现原理、与基于docker实现的playground的比较、与基于web server work的playground的比较、web container提供的功能与api等内容。

\vspace{\baselineskip}\noindent
\textsf{关键词:}Web Container;前端教学;教学站点;交互式验证

\clearpage
\section*{\textbf{ABSTRACT}}

With the continuous development of Internet technology, various technologies and tools continue to emerge, and the cost for developers to learn technologies and tools is getting higher and higher; from the initial server-side rendering to the current client-side rendering, that is, isomorphic rendering, people have changed from initially learning a language to now learning a whole set of tool chains, and developers are beginning to be tired of learning new frameworks and technologies.

This graduation project aims to design and implement a front-end teaching site based on web container, which is convenient for teachers to reproduce the development process step by step using documents and wizards. At the same time, students can simply modify the sample code and conduct interactive verification. The whole process does not require the establishment of any backend. This article focuses on the implementation and application of Web Container, and introduces in detail the implementation principle of Web Cotainer, the comparison with the playground based on docker, the comparison with the playground based on web server work, the functions and APIs provided by web container, etc.

\vspace{\baselineskip}\noindent
\textbf{Key words:} Web Container; Front-end teaching; Teaching site; Interactive verification

\clearpage
% 生成目录
\tableofcontents

\clearpage
% 将页码重置为 1
\pagenumbering{arabic}
% 开启页脚
\pagestyle{HDU-bachelor}

\section{绪论}

\subsection{国内外研究}

2021年5月12日,stackblitz 团队在 谷歌 I/O 上提到了以下的内容:

“几年前,我们就能感觉到,web 开发正在走向一个关键的拐点。WebAssembly 和新的 capabilities API 的出现,使得编写一个基于 WebAssembly 的操作系统似乎成为可能,该操作系统功能强大到可以完全在浏览器中运行 Node.js。 它提供一个比本地环境更快、更安全、更一致的卓越开发环境,以实现无缝代码协作,而无需设置本地环境,这个目标似乎离 Web 开发人员越来越近了。”

于是在同年,StackBlitz团队便推出了WebContainer,正如StackBlitz团队所提到的,其为浏览器提供了安全、稳定的Node.js运行时。目前国外的团队与公司还在持续优化WebContainer的性能、稳定性和安全性,并将WebContainer技术应用到在线教育平台和协作工具开发甚至AI技术中。

而在国内,尽管WebContainer技术的起步较晚,但近年来已经逐渐引起了学术界和工业界的关注,但相较于国外对于该技术的研究仍然相差甚远。一些前沿的技术团队和高校正开始着手研究WebContainer技术的实现原理及其潜在应用。他们试图通过解析WebContainer技术的核心机制,探索其在计算机技术教学、在线教育以及协作办公等领域的应用前景。


\subsection{课题的意义}

\begin{enumerate}

    \item 降低开发者学习成本:随着前端技术的快速发展,开发者需要学习的工具链和框架越来越多,学习成本也随之增加。通过基于WebContainer技术构建的前端教学站点,开发者可以在浏览器中直接运行和调试代码,无需搭建复杂的本地环境,从而降低了学习门槛和成本。

    \item 提升教学效率:传统的教学方式通常需要学生在本地环境中配置开发环境,这不仅耗时,还容易因环境差异导致问题。基于WebContainer的教学站点可以提供一个一致且安全的开发环境,学生可以直接在浏览器中修改和运行代码,实时查看结果,极大地提升了教学效率和学生的学习体验。

    \item 探索前沿技术应用:WebContainer技术基于WebAssembly(WASM),能够在浏览器中运行Node.js代码,这为前端开发带来了新的可能性。通过研究和实现基于WebContainer的教学站点,可以深入探索WASM、WebContainer等前沿技术的应用场景和潜力,推动这些技术在实际项目中的落地。

    \item 推动在线教育平台的发展:在线教育平台通常需要处理大量的用户请求和代码执行任务,传统的服务端执行方式可能会带来性能瓶颈和安全风险。WebContainer技术可以在客户端直接执行代码,减少对服务端的依赖,提升系统的性能和安全性。通过该课题的研究,可以为在线教育平台的技术架构提供新的思路和解决方案。

    \item 促进技术社区的发展:通过开源和分享基于WebContainer的教学站点,可以为技术社区提供一个学习和交流的平台,促进开发者之间的知识共享和技术进步。同时,该课题的研究成果也可以为其他开发者提供参考,推动WebContainer技术在前端开发中的广泛应用。
    

\end{enumerate}


\clearpage
\section{正文}



\clearpage
\section{结论}

\clearpage
\unnumberedsection{致谢}{致\hspace{2em}谢}

感谢TV。感谢所有TV,MTV。感谢广播。

今天,这里蓬荜生辉,人山人海,海枯石烂。我做梦都没想到我能成为这个作者。其实,我不是火命,\sout{我是水货},我是水命。

我呢,在这里,我感谢我的老伴。没有我的老伴就没有我的今天,因为我是陪她练的,把她练下去了,把我练上来了。俗话说,一个成功的男人,\sout{后背背一个多事的女人},背后必须有一个管事的女的,来管我。

我在这里,今天,感谢政府,\sout{能给我重新做人的机会},给我当作者的机会。我一定要\sout{坦白},坦率做人,坦诚做事。

\clearpage
% 修正目录超链接问题
\phantomsection{}
% 列出参考文献
\printbibliography[heading=bibintoc]

\clearpage
\unnumberedsection{附录}{附\hspace{2em}录}

\end{document}
